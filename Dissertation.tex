% This document is designed to conform to the style and layout
% guidelines stipulated in the document "Preparing and Filing the
% Thesis or Dissertation" at
% 
% http://www.gradstudies.ucdavis.edu/students/filing.html
% 
% and in the sample dissertation title page at
% 
% http://www.gradstudies.ucdavis.edu/students/sample_title.html
% 
% ---Tyrrell McAllister
%

% First we define the point size of the text as a variable because 
% we want some other variables to depend upon it.
%
\newcommand{\pointsize}{11pt}

\documentclass[oneside, \pointsize]{amsbook}

% Set the margins with the geometry package.  For the top margin,
% we have a half-inch to the header containing the page number.
% The remaining half-inch to the text will be introduced by our
% definitions of \headwidth and \headsep.
%
\usepackage[
   includehead,
   includefoot,
     left = 1.5in, 
      top = 0.5in, 
    right = 1in,
   bottom = 1in
]{geometry}
\usepackage{fancyhdr}
\usepackage{setspace}
\usepackage{calc}

% Set \headheight and \headsep so that \headheight + \headsep =
% 0.5in.  Thus, the top of the text will be one inch from the top
% of the page.
%
\setlength{\headheight}{\pointsize + 2pt}
\setlength{\headsep}{0.5in - \headheight} 

% Protrude page number half-inch into right margin so that it is a
% half-inch from the page's edge.
%
\fancyheadoffset[R]{0.5in} 

% The Preliminary Pages are to be numbered with small Roman
% Numerals that are centered at the bottom of the page.
%
\fancypagestyle{prelim}{%    
   \renewcommand{\headrulewidth}{0pt} 
   \fancyhf{}           
   \pagenumbering{roman}    
   \cfoot{-\thepage-}       
}

% Pages of the main text are to be numbered with arabic numerals
% that are in the upper right corner of the page.
%
% There are a couple additions to the header here that are not 
% stipulated in the dissertation guidelines.
%
%   (1) A headrule is added by setting the command \headrulewidth 
%   to be 0.4pt.
%
%   (2) The command \fancyhead[L]{\rightmark} causes the current
%   section to be indicated in the upper left of the page.  See 
%   documentation for the fancyhdr to control this display.
%
\fancypagestyle{maintext}{%
   \renewcommand{\headrulewidth}{0.4pt}
   \pagenumbering{arabic}
   \fancyhf{}
   \fancyhead[L]{\rightmark}
   \rhead{\thepage}
}

% Number figures, tables, and equations so that the chapter is
% included in the number.  E.g., use Figure 2.3 for the third
% figure in Chapter 2.
%
\numberwithin{figure}{chapter} 
\numberwithin{table}{chapter}
\numberwithin{equation}{chapter}
\numberwithin{section}{chapter}


% Use this file to load additional packages or define macro
% commands that will be used in the dissertation.
%

\usepackage{amsmath,amsfonts}
\usepackage{amsthm}
\usepackage{geometry}
\usepackage{pgf}
\usepackage{xspace}
\usepackage{hyperref}
\usepackage{framed}
\usepackage{graphicx}
\usepackage{caption}
\usepackage{subcaption}
\usepackage{setspace} % for double space

\usepackage{algorithm}
\usepackage{algorithmic}  
\renewcommand{\algorithmicrequire}{\textbf{Input:}}
\renewcommand{\algorithmicensure}{\textbf{Output:}}


\newtheorem{theorem}{Theorem}[section]
\newtheorem{lemma}[theorem]{Lemma}
\newtheorem{conjecture}[theorem]{Conjecture}
\newtheorem{claim}[theorem]{Claim}
\newtheorem{corollary}[theorem]{Corollary}
\newtheorem{cor}[theorem]{Corollary}
\newtheorem{proposition}[theorem]{Proposition}
\newtheorem{prop}[theorem]{Proposition}
\theoremstyle{definition}
\newtheorem{problem}[theorem]{Problem}
\newtheorem{definition}[theorem]{Definition}

\newtheorem{remark}[theorem]{Remark}
\newtheorem{example}[theorem]{Example}


\newcommand{\fmax}{f_{\text{\rm max}}}
\newcommand{\fmin}{f_{\text{\rm min}}}
\newcommand{\fhan}{f_{\text{\rm han}}}

\newcommand{\C}{{\mathbb C}}
\newcommand{\R}{{\mathbb R}}
\newcommand{\Z}{{\mathbb Z}}
\newcommand{\Q}{{\mathbb Q}}	


\let\ve=\mathbf
\newcommand\fractional[1]{\{#1\}}
\newcommand\ceil[1]{\lceil{#1}\rceil}
\newcommand\floor[1]{\lfloor{#1}\rfloor}
\renewcommand{\ll}{{\langle}}
\newcommand{\rr}{{\rangle}}
\newcommand{\reg}{{\mathrm{reg}}}
\newcommand{\CR}{{\mathcal{B}}}
\newcommand{\CG}{{\mathcal{G}}}
\newcommand{\CE}{{\mathcal E}}
\newcommand{\CF}{{\mathcal F}}

\newcommand{\PP}{{ P}}

\newcommand{\Res}{\operatorname{Res}}
\newcommand{\res}{\operatorname{res}}
\newcommand{\Top}{\operatorname{Top}}
\newcommand{\s}{\mathfrak{s}}
\newcommand{\Ber}{\operatorname{Ber}}
\newcommand{\Cone}{\operatorname{Cone}}
\newcommand{\TCone}{\operatorname{TCone}}
\newcommand{\Todd}{\operatorname{Todd}}
\newcommand{\vol}{{\mathrm{\rm vol}}}
\renewcommand{\ll}{{\langle}}
\newcommand{\rrr}{{\rangle}}
\newcommand{\aff}{{\mathrm{\rm aff}}}
\newcommand{\rhat}{\hat R(T)}
\newcommand\coneC{\mathfrak{c}}
\newcommand\coneW{\mathfrak{w}}
\newcommand\coneU{\mathfrak{u}}

\DeclareMathOperator{\lcm}{lcm}
\renewcommand\d{\mathrm d}
\newcommand\e{\mathrm e}
\newcommand\T{\top}

\newcommand\vexi{\boldsymbol{\xi}}
\newcommand\vebeta{\boldsymbol{\beta}}
\newcommand\veeta{\boldsymbol{\eta}}

\newcommand{\norm}[1]{\left\lVert#1\right\rVert}
\renewcommand\d{\,\mathrm{d}}


\newcommand\latteintegrale{{\tt LattE integrale}\xspace} 
\newcommand\maple{{\tt Maple}\xspace}
\newcommand\mapleKnapsack{\emph{M-Knapsack}\xspace}
\newcommand\latteKnapsack{\emph{LattE Knapsack}\xspace}
\newcommand\coneApx{\emph{LattE Top-Ehrhart}\xspace}

\newcommand\maplecode[1]{\textsf{#1}}
\newcommand\shellcode[1]{\texttt{#1}}

\newcommand{\tmagenta}[1]{\small \textsf{\textcolor {magenta} {#1}}} % TODO item, no input needed
\newcommand{\tgreen}[1]{\small \textsf{\textcolor {ForestGreen} {#1}}} % resolved question; no action needed
\newcommand{\tred}[1]{\small \texttt{\textcolor {red} {#1}}} % open question, input requested
\newcommand{\tblue}[1]{\small \textsf{\textcolor {blue} {#1}}} % answer to question or comment


\newcommand\vealpha{\boldsymbol{\alpha}}
\renewcommand{\a}{{\vealpha}}




\begin{document}
   \frontmatter

   \pagestyle{prelim}
   
   % Redefine plain page style so that the first pages of chapters
   % have desired page style.
   %
   \fancypagestyle{plain}{%
      \fancyhf{}
      \cfoot{-\thepage-}
   }%
   \begin{center}
   \null\vfill
   \textbf{%
     Decomposition methods for nonlinear optimization and data mining
   }%
   \\
   \bigskip
   By \\
   \bigskip
   BRANDON E. DUTRA \\
   \bigskip
   B.S. (University of California, Davis) 2012 \\
   \bigskip
   DISSERTATION \\
   \bigskip
   Submitted in partial satisfaction of the requirements for the
   degree of \\
   \bigskip
   DOCTOR OF PHILOSOPHY \\
   \bigskip
   in \\
   \bigskip
   Applied Mathematics \\
   \bigskip
   in the \\
   \bigskip
   OFFICE OF GRADUATE STUDIES \\
   \bigskip        
   of the \\
   \bigskip
   UNIVERSITY OF CALIFORNIA \\
   \bigskip
   DAVIS \\
   \bigskip
   Approved: \\
   \bigskip
   \bigskip
   \makebox[3in]{\hrulefill} \\
   Committee Member 1 \\
   \bigskip
   \bigskip
   \makebox[3in]{\hrulefill} \\
   Committee Member 2 \\
   \bigskip
   \bigskip
   \makebox[3in]{\hrulefill} \\
   Committee Member 3 \\
   \bigskip
   Committee in Charge \\
   \bigskip
   2016 \\
   \vfill
\end{center}

   \newpage
   
   % Begin Double Spacing
   %
%   \doublespacing
   
   \tableofcontents
   \newpage
   
   This Abstract Found
   \newpage
   
   \section*{Acknowledgments}
   I'd like to thank the little people.
   
   \mainmatter
   
   \pagestyle{maintext}
   
   % Redefine plain page style so that the first pages of 
   % chapters have desired page style.
   %
   \fancypagestyle{plain}{%
      \renewcommand{\headrulewidth}{0pt}
      \fancyhf{}
      \rhead{\thepage}
   }%
   
   \chapter{Introduction}
   \label{ch:IntroductionLabel}
   Allow me to introduce you to my dissertation.
   
   %%%%%%%%%%%%%%%%%%%%%%
   \chapter{Background}
   \label{ch:background}
   %
this is cool   
   
   In this chapter, we first review some polyhedral decompositions and how these relate to generating functions for the lattice points of a polyhedra. 
   
\section{Working with generating functions: an example}
 
 Let us start with an easy example. Consider the one dimensional polyhedra in $\R$ given by $\PP = [0, n]$. We encode the latticd points of $\PP \cap \Z$ by placing each integer point as the power of a monomial, there by obtaining the polynomial $S(\PP; z) := z^0 + z + z^2 + z^3 + \cdots + z^n$. The polynomial $S(\PP; z)$ is called the \emph{generating function of $\PP$.} Notice that counting $\PP \cap \Z$ is equivalat to evaluating $S(\PP, 1)$.
 
 In terms of the computational complexity, listing each monomial in the polynomial $S(\PP, z)$ results in a polynomial with exponential length in the bit length of $n$. However, we can rewrite the summation with one term:
 
 \[ S(\PP, z) = z^0 + z^1 + \cdots + z^n = \frac{1-z^{n+1}}{1-z}.\]

Counting the number of point in $|\PP \cap \Z|$ is no longer as simple as evaluating $\frac{1-z^{n+1}}{1-z}$ at $z=1$ because this is a singularity. However, this singularity is removable. One could perform long-polynomial division, but this would result in a expoentially long polynomial in the bit length of $n$. Another option that yeilds a polynomail time algorithm would be to apply L'Hospital's rule:

\[\lim_{z \rightarrow 1} S(\PP, z) = \lim_{z \rightarrow 1} \frac{-(n+1)z^{n}}{1} = n+1. \]

Notice that $S(\PP, z)$ can be written in two ways:

\[ S(\PP, z) = \frac{1}{1-z} - \frac{z^{n+1}}{1-z} = \frac{1}{1-z} + \frac{z^n}{1-z^{-1}}.\]

The first two rational expressions have a nice description in terms of their series expansion:

\[ 1+z + \cdots + z^n = (z^0 + z^1 + \cdots ) -  (z^{n+1} + z^{n+2} + \cdots).\]

For the secont two rational functions, we have to be careful about the domain of convergence when computing the series expansion. Notice that in the series expansion,

\begin{align*}
\frac{1}{1-z} =& \begin{cases} 
      z^0 + z^1 + z^2 \cdots & \;\;\;\;\; \text{ if } |z| < 1 \\
      -z^{-1} -z^{-2} - z^{-3} - \cdots & \;\;\;\;\; \text{ if }  |z| > 1 
   \end{cases} \\
\frac{z^n}{1-z^{-1}} =& \begin{cases} 
		-z^{n+1} - z^{n+2} - z^{n+3} -\cdots & \text{ if } |z| < 1 \\
      z^{n} + z^{n-1} + z^{n-2} + \cdots & \text{ if }  |z| > 1 
   \end{cases} \\
\end{align*}


adding the terms when $|z| <1$  or $|z| > 1$ results in the desired polynomial: $z^0 + z^1 + \cdots + z^n$. But we can also get the correct polynomial by adding the series that corresponds to different domains of convergence. However, to do this we must now add the series $\cdots + z^{-2} + z^{-1} + 0 + z + z^2 + \cdots$ which corresponds to the polyhedra that is the entire real line:
\begin{align*}
1 + z + \cdots z^n &= (1 + z + z^2 + \cdots)\\
& \quad  + (z^n + z^{n-1} + \cdots)  \\
& \quad - (\cdots + z^{-2} + z^{-1} + 0 + z + z^2 + \cdots)
\end{align*}
and
\begin{align*}
1 + z + \cdots z^n &= (-z^{-1} -z^{-2} - z^{-3} - \cdots)\\
& \quad  + (-z^{n+1} - z^{n+2} - z^{n+3} -\cdots)  \\
& \quad + (\cdots + z^{-2} + z^{-1} + 0 + z + z^2 + \cdots)
\end{align*}

Hence by including the series $\cdots + z^{-2} + z^{-1} + 0 + z + z^2 + \cdots$, we can perform the series expansion of $\frac{1}{1-z} + \frac{z^n}{1-z^{-1}}$ by computing the series expansion of each term on potentially different domains of convergence. 

In the next sections, we will develop rigorous justification for adding the series  $\cdots + z^{-2} + z^{-1} + 0 + z + z^2 + \cdots$.
 
\section{Indicator functions}

\begin{definition} 
The indicator function, $[A]: \R^d \rightarrow \R$, of a set $A \subseteq \R^d$ takes two values: $[A](x) =1$ if $x \in A$ and $[A](x)=0$ otherwise. 
\end{definition} 

The set of indicator functions on $\R^d$ froms a vectorspace with pointwise additions and scalar multiplication. Notice that $[A]\cdot [B] = [A \cap B]$, and $[A]+[B] = [A \cup B] + [A \cap B]$. 

\begin{definition} The \emph{cone} of a set $A \subseteq \R^d$ is all conic combinations of the points from $A$:
\[\Cone(A) := \left\{ \sum_i \alpha_i a_i \mid a_i \in A, \alpha_i \in \R_{\geq 0} \right\} \]
\end{definition} 

\begin{definition} 
Let $\PP$ be a polyhedron and $x \in \PP$. Then the \emph{tangent cone}, of $\PP$ at $x$ is the polyhedral cone 
\[\TCone(\PP, x) := x + \Cone(\PP - x)\]  
\end{definition} 
 
Note that if $x$ is a vertex of $\PP$, and $\PP$ is given by an inequality description, then the tangent cone $\TCone(\PP, x)$ is the intersection of inequalities that are tight at $x$. Also, $\TCone(\PP, x)$ includes the affine hull of the face that $x$ is in, so the tangent cone is pointed only if $x$ is a vertex. 

When $F$ is a face of $\PP$, we will also use the notation $\TCone(\PP, F)$ to denote $\TCone(\PP, x)$ where $x$ is any interior point of $F$.

\begin{theorem}[\cite{brianchon1837}, \cite{gram1871}]
Let $\PP$ be a polyhedron, then 
\[[\PP] = \sum_{F} (-1)^{\dim(F)}[\TCone(\PP, F)] \]
where the sum ranges over all faces $F$ of $\PP$ including $F = \PP$ but excluding $F = \emptyset$
\end{theorem} 

 
   \begin{enumerate}
   \item ex of a line via polynomial
   \item 2d tile case
   \item Brion, tangent cones
   \item Barvinok
   
   \end{enumerate}
   
   
   
   
   
   
   
   
   
   %%%%%%%%%%%%%%%%%%%%%%%%%%%%%

   \chapter[% 
      Short Title of 2nd Ch.
   ]{% 
      Long Title of Second Chapter
   }%
   \label{ch:2ndChapterLabel}
   Rain is wet.  The conclusions are immediate and self-evident.  We
leave them as an exercise for the reader.
   
   \appendix

   \chapter[%
      Short Title of Appendix A
   ]{%
      Long Title of Appendix A
   }%
   \label{ch:AppendixALabel}
   Observations of non-wet rain have recently appeared in the
literature.  In this Appendix, we briefly consider the
implications of these observations for the analysis offered in
this dissertation.

       
   \backmatter
   
   \bibliographystyle{amsalpha-fi-arxlast}
   \bibliography{bibliography/biblio}
\end{document}